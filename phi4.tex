\section{$\varphi^4$理論}

\subsection{自由な実スカラー場の理論の2点Green関数}
この節では、ラグランジアン密度
\begin{equation}
  \calL=\frac{1}{2}\bigl(\partial^\mu\varphi\partial_\mu\varphi-m^2\varphi\bigr)
\end{equation}
($m$は正の定数)
により定まる場の理論の2点Green関数を求める。
この理論の運動方程式は、Klein-Gordon方程式
\begin{equation}
  (\partial^2+m^2)\varphi=0
\end{equation}
である。
また、ハミルトニアン密度は
\begin{equation}
  \calH=\frac{1}{2}\bigl(\partial^\mu\varphi\partial_\mu\varphi+m^2\varphi\bigr)
\end{equation}
である。

ある時刻$t$における場の演算子$\bfx \lmt \varphi(t,\bfx)$のFourier変換
$\hat{\varphi}(t;\bfk)$を用いて、
$\varphi(x)$を
\begin{equation}
  \varphi(t,\bfx)=
  \frac{1}{\sqrt{(2\pi)^3}}
  \int d^3\bfk\hat{\varphi}(t;\bfk) e^{i\bfk\cdot\bfx}
\end{equation}
と表示する。
$\varphi(x)$がエルミートであることは
$\hat{\varphi}(t;\bfk)^\dagger=\hat{\varphi}(t;-\bfk)$
と同値であることに注意する。
$E(\bfk)=\sqrt{\bfk^2+m^2}$とおく。
場の演算子の運動方程式$(\partial^2+m^2)\varphi=0$より、
Fourier変換$\hat{\varphi}(t;\bfk)$についての運動方程式
\begin{equation}
  \Bigl(\frac{d^2}{dt^2} - (i\bfk)^2 + m^2\Bigr)\hat{\varphi}(t;\bfk) = 0
  \iff
  \frac{d^2\hat{\varphi}(t;\bfk)}{dt^2} = -E(\bfk)^2\hat{\varphi}(t;\bfk)
\end{equation}
が得られる。
この微分方程式の一般解は
\begin{equation}
  \hat{\varphi}(t;\bfk)=
  A(\bfk)e^{-iE(\bfk)t}
  +
  B(\bfk)e^{iE(\bfk)t}
\end{equation}
であり、
条件$\hat{\varphi}(t;\bfk)^\dagger=\hat{\varphi}(t;-\bfk)$は
$B(\bfk)^\dagger=A(-\bfk)$と同値である。

よって
\begin{align*}
  \varphi\bigl(t,\bfx\bigr) &=
  \frac{1}{\sqrt{(2\pi)^3}}
  \int d^3\bfk\hat{\varphi}(t;\bfk) e^{i\bfk\cdot\bfx} \\ &=
  \frac{1}{\sqrt{(2\pi)^3}}
  \int d^3\bfk
  \bigl( A(\bfk)e^{-iE(\bfk)t}+B(\bfk)e^{iE(\bfk)t} \bigr)e^{i\bfk\cdot\bfx} \\ &=
  \frac{1}{\sqrt{(2\pi)^3}}
  \int d^3\bfk
  \bigl(
  A(\bfk)e^{-i(E(\bfk)t-\bfk\cdot\bfx)}
  +B(-\bfk)e^{i(E(\bfk)t-\bfk\bfx)}
  \bigr) \\ &=
  \frac{1}{\sqrt{(2\pi)^3}}
  \int d^3\bfk
  \bigl(
  A(\bfk)e^{-i(E(\bfk)t-\bfk\cdot\bfx)}
  +A(\bfk)^\dagger e^{i(E(\bfk)t-\bfk\bfx)}
  \bigr)
  \\ &=
 \frac{1}{\sqrt{(2\pi)^3}}
 \int d^3\bfk
 \bigl.\bigl(
 A(\bfk)e^{-ikx}
 +A(\bfk)^\dagger e^{ikx}
 \bigr)\bigr|_{k_0=E(\bfk)}
\end{align*}
と表示することができる。
これは、
$\varphi(x)$がエルミートであるようなクライン・ゴルドン方程式の一般解である。

ここで、消滅演算子$a(\bfk)$を
\begin{equation}
  \frac{a(\bfk)}{\sqrt{2E(\bfk)}}=A(\bfk)
\end{equation}
により定めれば、
\begin{equation}
  \varphi(x) =
  \int
  \frac{d^3\bfk}{\sqrt{(2\pi)^3 2E(\bfk)}}
  \bigl. \bigl(
    a(\bfk)e^{-ikx}+a^\dagger(\bfk)e^{ikx}
  \bigr) \bigr|_{k_0=E(\bfk)}
\end{equation}
となる。

この表示を用いて、
複素時刻$t\in\bbC$に対する$\varphi(t,\bfx)$の値を定義する。

\begin{thm}
  \label{thm-residue-energy-pole}
  数$E \in\bbR_{>0}, \tau\in e^{-i\delta}\bbR$に対し、積分
  \begin{equation}
    I(\tau,E)=\int_{-e^{i\delta}\infty}^{e^{i\delta}\infty}
    \frac{e^{-i\omega\tau}}{\omega^2-E^2}
    d\omega
    \quad
    (\omega\in e^{i\delta}\bbR)
  \end{equation}
  を計算すると、
  \begin{equation}
    I(\tau,E)=
    \begin{cases}
      \frac{\pi i}{E} e^{-iE\tau} & (\tau > 0)\\
      \frac{\pi i}{E} e^{iE\tau} & (\tau < 0)
    \end{cases}
  \end{equation}
  となる。
\end{thm}

\begin{proof}
留数定理。
\end{proof}

\begin{thm}
  $\delta \in (0,2\pi)$と
  $x,y \in e^{-i\delta}\bbR \times \bbR^3$に対し、
  \begin{equation}
  iG_\delta(x,y)=
  \int_{k_0\in e^{-i\delta}\bbR}
  \frac{d^4k}{(2\pi)^4}
  \frac{e^{-ik(x-y)}}{k^2-m^2}
  \end{equation}
  が成り立つ。
\end{thm}

\begin{proof}
  グリーン関数
  \begin{equation}
    G_\delta(x,y)=\bra{0}\calT\{\varphi(x)\varphi(y)\}\ket{0}
  \end{equation}
  を計算する。
  $x^0>y^0$のとき、$a(\bfk)\ket{0}=0$より
  \begin{equation}
    \bra{0}\varphi(x)\varphi(y)\ket{0}=
    \int \frac{d^3\bfk d^3\bfk'}{(2\pi)^3 2\sqrt{E(\bfk)E(\bfk')}}
    \bra{0}a(\bfk)a^\dagger(\bfk')\ket{0}
    \bigl.e^{-ikx}e^{ik'y}\bigr|_{k_0=E(\bfk),k'_0=E(\bfk')}
  \end{equation}
  である。
  交換関係
  $[a(\bfk),a^\dagger(\bfk')]=\delta^3(\bfk-\bfk')$
  より$\bra{0}a(\bfk)a^\dagger(\bfk')\ket{0}=\delta^3(\bfk-\bfk')$
  となるため、
  \begin{equation}
    \bra{0}\varphi(x)\varphi(y)\ket{0}=
    \int \frac{d^3\bfk}{2(2\pi)^3E(\bfk)}
    e^{-ik(x-y)}\bigr|_{k_0=E(\bfk)}
  \end{equation}
  が得られる。
  よって、一般の$x,y$に対し、
  \begin{equation}
    G_\delta(x,y)=
    \int \frac{d^3\bfk}{(2\pi)^3 2E(\bfk)}
    \bigl.
    \bigl(
    \theta(x^0-y^0)
    e^{-ik(x-y)}
    +
    \theta(y^0-x^0)
    e^{ik(x-y)}
    \bigr)
    \bigr|_{k_0=E(\bfk)}
  \end{equation}
  である。
  定理\ref{thm-residue-energy-pole}を用いると、
  \begin{align*}
    G_\delta(x,y) &=
    \int \frac{d^3\bfk}{(2\pi)^3 4\pi i}
    \Bigl(
    \theta(x^0-y^0)
    I\bigl(x^0-y^0,E(\bfk)\bigr)
    e^{-i\bfk(\bfx-\bfy)}
    +
    \theta(y^0-x^0)
    I\bigl(x^0-y^0,E(\bfk)\bigr)
    e^{i\bfk(\bfx-\bfy)}
    \Bigr) \\  &=
    \int \frac{d^3\bfk}{(2\pi)^3 4\pi i}
    \Bigl(
    \theta(x^0-y^0)
    I\bigl(x^0-y^0,E(\bfk)\bigr)
    e^{-i\bfk(\bfx-\bfy)}
    +
    \theta(y^0-x^0)
    I\bigl(x^0-y^0,E(\bfk)\bigr)
    e^{-i\bfk(\bfx-\bfy)}
    \Bigr) \\ &=
    \int \frac{d^3\bfk}{(2\pi)^3 4\pi i}
    \bigl(\theta(x^0-y^0)+\theta(y^0-x^0)\bigr)
    I\bigl(x^0-y^0,E(\bfk)\bigr)
    e^{-i\bfk(\bfx-\bfy)} \\ &=
    \int \frac{d^3\bfk}{(2\pi)^3 2\pi i}
    I\bigl(x^0-y^0,E(\bfk)\bigr)
    e^{-i\bfk(\bfx-\bfy)} \\ &=
    \int \frac{d^3\bfk}{(2\pi)^3 2\pi i}
    \int_{-e^{i\delta}\infty}^{e^{i\delta}\infty}
    dk_0
    \frac{e^{-ik_0(x^0-y^0)}}{k_0^2-E(\bfk)^2}
    e^{-i\bfk(\bfx-\bfy)} \\ &=
    \int \frac{d^3\bfk}{(2\pi)^3 2\pi i}
    \int_{-e^{i\delta}\infty}^{e^{i\delta}\infty}
    dk_0
    \frac{e^{-ik(x-y)}}{k^2-m^2}
  \end{align*}
  より主張を得る。
\end{proof}
