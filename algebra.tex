\section{代数学}

\subsection{クリフォード代数}
$K$を体$\bbR$または体$\bbC$とする。
$V$を有限次元$K$ベクトル空間とする。
$\langle,\rangle \colon V \otimes V \lra K$
を対称双線型形式とする。

\begin{dfn}
  \label{definition-of-clifford-algebra}
   クリフォード代数$\Cl(V,\langle,\rangle)$とは、
   組$(A, i)$であって、条件
   \begin{enumerate}
     \item
     $A$は$K$上の代数。
     \item
     $i \colon V \lra A$は線形写像で、
     任意の$x, y \in V$に対し関係式
     \begin{equation}
       \label{clifford-algebra-defining-relation}
       \{i(x),i(y)\} = 2\langle x,y\rangle
     \end{equation}
     が成り立つ。
   \end{enumerate}
   ようなもものうち、普遍的なもののことである。

   明示的には, $V$が生成するテンソル代数
   \begin{equation}
     T(V) = \bigoplus_{n\geq 0}V^{\otimes n}
   \end{equation}
   を考え、自然な単射写像を$i \colon V \lra T(V)$と書くとき、
   関係式(\ref{clifford-algebra-defining-relation})
   で$T(V)$を剰余して得られる代数が$\Cl(V,\langle,\rangle)$である。
\end{dfn}

\begin{rem}
  定義(\ref{definition-of-clifford-algebra})の状況を考える。
  テンソル代数$T(V)$は自然な$\bbZ_2 = \bbZ/2\bbZ$次数付けを持つ。
  関係式(\ref{clifford-algebra-defining-relation})
  が生成するイデアルはこの次数について斉次であるため、
  Clifford代数$\Cl(V,\langle,\rangle)$も$\bbZ_2$次数付けを持つ。
  次数$m$の部分空間($m=0,1$)は$\Cl(V,\langle,\rangle)_m$と表記される。
\end{rem}

\begin{rem}
$i \colon V \lra \Cl(V,\langle,\rangle)$が単射であることが次のようにしてわかる。
関係式(\ref{clifford-algebra-defining-relation})が生成する$T(V)$のイデアルを$I$とする。
また、イデアル$\bigoplus_{n\geq 1}V^{\otimes n} \subset T(V)$を$J$とする。
$I \subset J$であるため, イデアル$J/I \subset \Cl(V,\langle,\rangle)$を考えることができる
よって、線形写像$j\colon\Cl(V,\langle,\rangle) \lra \Cl(V,\langle,\rangle)/(J/I) = T(V)/J = V$
が得られる。
この線形写像が$j \circ i = \id$を満たすので、$i$は単射である。
以下では$V \subset \Cl(V,\langle,\rangle)$と考え、
記号$i$は省略する。
\end{rem}

\begin{dfn}
  線型写像$V\lra V; x \lmt -x$は、
  $K$代数の準同型
  $\alpha\colon\Cl(V,\langle,\rangle)\lra\Cl(V,\langle,\rangle)$
  を誘導する。
\end{dfn}

\begin{dfn}
  \label{dfn-beta-tranposing-antihomomorphism-clifford-algebra}
  $K$代数の反準同型
  \begin{equation}
    \beta \colon T(V)\lra T(V);
    x_1\otimes\cdots\otimes x_n \lra
    x_n\otimes\cdots\otimes x_1
  \end{equation}
  は、$K$代数の反準同型
  $\beta \colon\Cl(V,\langle,\rangle)\lra\Cl(V,\langle,\rangle)$
  を誘導する。
\end{dfn}

\begin{dfn}
  $p,q\geq 0$を非負整数とする。
  $\Cl^{p+q}(\bbR)$は、$\bbR^{p+q}$上の双線型形式
  \begin{equation}
    \langle(x_1,\ldots,x_{p+q}),(y_1,\ldots,y_{p+q})\rangle=
    \sum_{i=1}^{p}x_iy_i-\sum_{i={p+1}}^{p+q}x_iy_i
  \end{equation}
  から定まるClifford代数のことである。
\end{dfn}

\subsection{パウリ行列}
\begin{dfn}
  Pauli行列 $\sigma_i \in \Mat_2(\bbC)$($i=1,2,3$)とは、
  \begin{equation}
    \sigma_1 = \begin{pmatrix}0&1\\1&0\end{pmatrix},\quad
    \sigma_2 = \begin{pmatrix}0&-i\\i&0\end{pmatrix},\quad
    \sigma_3 = \begin{pmatrix}1&0\\0&-1\end{pmatrix}
  \end{equation}
  のことである。
\end{dfn}

\begin{thm}
  次の関係式が成り立つ。
  \begin{equation}
    \sigma_i^2 = 1\ (i = 1,2,3),\quad
    \sigma_i\sigma_j=-\sigma_j\sigma_i=i\sigma_k
    \ \bigl((i,j,k)=(1,2,3),(2,3,1),(3,1,2)\bigr).
  \end{equation}
  \begin{equation}
    \{\sigma_i,\sigma_j\}=2\delta_{ij}.
  \end{equation}
  \begin{equation}
    \det \sigma_i = -1,\quad \Tr \sigma_i = 0, \quad
    \Tr (\sigma_i \sigma_j) = 2\delta_{ij}.
  \end{equation}
\end{thm}

\begin{thm}
  \label{I-and-Pauli-matrices-is-ortho-basis}
  \begin{enumerate}
    \item 行列$\{I, \sigma_1, \sigma_2, \sigma_3\}$は、
          Hilbert-Smidt内積について互いに直交かつそれぞれの長さが$\sqrt{2}$である。
    \item 行列$\{\sigma_1, \sigma_2, \sigma_3\}$は、
          トレースのないHermite行列のなす$\bbR$線型空間
          $\{M\in\Mat_2(\bbC) \mid M^\dagger=M, \Tr M = 0\}$
          の基底である。
    \item 行列$\{I, \sigma_1, \sigma_2, \sigma_3\}$は、
          Hermite行列全体のなす$\bbR$線型空間$\{M\in\Mat_2(\bbC) \mid M^\dagger=M\}$
          の基底である。
    \item 行列$\{I, \sigma_1, \sigma_2, \sigma_3\}$は、
          $\bbC$線型空間$\Mat_2(\bbC)$の基底である。
  \end{enumerate}
\end{thm}

\begin{thm}
  $e_i \in \bbR^{3}$を標準基底とする($i=1,2,3$)。
  Pauli行列が生成する$\Mat_2(\bbC)$の部分$\bbR$代数$A$は$\Mat_2(\bbC)$全体である。
  また、自然な同型
  \begin{equation}
    \Cl^{3+0}(\bbR) \isolra A = \Mat_2(\bbC); e_i \lmt \sigma_i
  \end{equation}
  がある。
\end{thm}

\begin{proof}
  Pauli行列の交換関係とClifford代数の普遍性より、
  全射$\Cl^{3+0}(\bbR)\lra A$が誘導される。
  これが全単射であることを示すためには$\dim A\geq\dim\Cl^{3+0}(\bbR)=8$を示せばよい。
  $I\in\Mat_2(\bbC)$を単位行列とするとき、
  \begin{equation}
    I, \sigma_1, \sigma_2, \sigma_3,
    iI, i\sigma_1, i\sigma_2, i\sigma_3 \in A
  \end{equation}
  である。
  定理\ref{I-and-Pauli-matrices-is-ortho-basis}より、
  \begin{equation}
  \Mat_2(\bbC)=\bbC I\oplus\bbC\sigma_1\oplus\bbC\sigma_2\oplus\bbC\sigma_3
  =\bbR I\oplus\bbR\sigma_1\oplus\bbR\sigma_2\oplus\bbR\sigma_3
  \oplus\bbR iI\oplus\bbR i\sigma_1\oplus\bbR i\sigma_2\oplus\bbR i\sigma_3
  \end{equation}
  なので、これらは$\bbR$上で$8$次元ベクトル空間を生成する。
\end{proof}

\subsection{ガンマ行列}
\begin{dfn}
  ガンマ行列$\gamma^{\mu} \in \Mat_4(\bbC)$($\mu=0,1,2,3$)とは、
  \begin{equation}
    \gamma^0 = \begin{pmatrix}I&0\\0&-I\end{pmatrix},\quad
    \gamma^i = \begin{pmatrix}0&\sigma_i\\-\sigma_i&0\end{pmatrix}
    \ (i=1,2,3)
  \end{equation}
  のことである。
\end{dfn}

\begin{thm}
  $\bbR$代数の準同型
  $\Cl^{1+3}(\bbR)\lra \Mat_4(\bbC);\ e_\mu \lmt \gamma^\mu$
  が定まる。
  さらに、これは$\bbC$代数の同型
  $\Cl^{1+3}(\bbC)\isolra \Mat_4(\bbC)$
  を誘導する。
\end{thm}

\begin{proof}
  Zeeの本の計算方法がわかりやすくて良い。
\end{proof}

\begin{dfn}
  Clifford代数$\Cl^{1+3}(\bbC)$には、
  同型を除いて唯一の既約複素表現$\Delta$が存在し、その次元は$4$である。
\end{dfn}

\begin{dfn}
  $e_5=ie_0e_1e_2e_3$により元$e_5\in\Cl^{1+3}(\bbC)$を定義する
  (「$e_4$」ではないことに注意)。
  また、$\bbC$代数の同型
  $\Cl^{1+3}(\bbC)\isolra \Mat_4(\bbC)$
  によって$e_5$に対応する行列を$\gamma^5\in \Mat_4(\bbC))$とする。
  具体的には
  \begin{equation}
    \gamma^5=\begin{pmatrix}0&I\\I&0\end{pmatrix}
  \end{equation}
  である。
\end{dfn}

\begin{rem}
  $\gamma^5$の計算はやはりZeeの本の方法で行うこと。
\end{rem}

\begin{thm}
  $(e_5)^2=1$が成り立つ。
\end{thm}

\begin{proof}
  Clifford代数の定義の関係式を用いて計算すると、
  \begin{equation}
    (e_5)^2=i^2(e_0)^2(e_1)^2(e_2)^2(e_3)^2
    =(-1)\cdot 1\cdot(-1)\cdot(-1)\cdot(-1)=1
  \end{equation}
  となる。
\end{proof}

\begin{thm}
  \label{gamma5-anticommute-gamma-matrices}
  反交換関係$\{e_5,e_\mu\}=0$($\mu=0,1,2,3$)が成り立つ。
\end{thm}

\begin{thm}
  Clifford代数$\Cl^{1+3}(\bbC)$の次数$0$部分(偶数部分)を$\Cl^{1+3}_\even(\bbC)$と書く。
  このとき、$e_5$は$\Cl^{1+3}_\even(\bbC)$の中心に属する。
\end{thm}

\begin{proof}
  定理\ref{gamma5-anticommute-gamma-matrices}から容易に従う。
\end{proof}

\begin{rem}
  $e_5$は$\Cl^{1+3}(\bbC)$の中心には属さない。
  これは、$\gamma^5$が対角行列ではないことからわかる。
\end{rem}

\begin{dfn}
  Clifford代数$\Cl^{1+3}(\bbC)$の既約表現$\Delta$を選ぶ。
  このとき、$e_5$による$\Delta$の固有空間分解を
  \begin{equation}
    \Sigma_{\pm}=\{ s \in \Delta \mid e_5 s = \pm s \}
  \end{equation}
  と定義する。
  射影作用素
  $P_{\pm}=(1\pm e_5)/2$
  を用いて
  $\Sigma_{\pm}=P_{\pm}\Delta$
  と定義することもできる。
  これに伴い、表現$\Delta$を$\Cl^{1+3}_\even(\bbC)$への制限は
  $2$つの表現$\Sigma_{\pm}$に分解する。
  これらの表現をWeyl表現、あるいはカイラル表現と呼ぶ。
\end{dfn}

\begin{rem}
  +/-の代わりにR/Lを用い、右手表現、左手表現と呼ぶこともある。
\end{rem}

\begin{thm}
  Wely表現は、偶数部分$\Cl^{1+3}_\even(\bbC)$の$2$つの相異なる既約表現であり、
  また、$\Cl^{1+3}_\even(\bbC)$の既約表現はこれらのみである。
\end{thm}

\begin{dfn}
  $V$を有限次元$\bbC$ベクトル空間とする。
  \begin{enumerate}
    \item
    複素共役準同型$\sigma\colon\bbC\lra\bbC$に沿って$V$を係数拡大したもの
    $V\otimes\sigma$を$\overline{V}$と書き、
    $V$の複素共役空間と呼ぶ。
    この操作は双対と交換する。
    すなわち、自然な同一視 $\overline{V^{\vee}}=\overline{V}^{\vee}$ が存在する。
    \item
    反線型写像$f \colon V\lra V$とは、$z \in \bbC, v \in V$
    に対して$f(zv)=\overline{z}v$を満たす$\bbR$線型写像のことである。
    言い換えれば、$\bbC$線型写像$\overline{V}\lra V$や$V\lra\overline{V}$を
    自然な包含写像$V\lra\overline{V}$によって
    写像$V\lra V$とみなしたもののことである。
    \item
    $V$上のHermitian形式とは、
    $\bbC$線型写像
    $\langle,\rangle\colon\overline{V}\otimes_{\bbC}V\lra\bbC$
    のことである。
    これは$\bbC$線型写像
    $V\lra\overline{V}^\vee; v \lmt (w \lmt \langle w, v\rangle)$ を誘導する。
    \item
    $\bbC$代数の反準同型写像
    \begin{equation}
      \End_{\bbC}(V)\lra\End_{\bbC}(V^{\vee});
      f \lmt (g \lmt g \circ f )
    \end{equation}
    を転置とよぶ。
    これは、$V$の基底とそれに伴う$V^{\vee}$の双対基底のもとで、
    表現行列の転置をとる操作に対応する。
    \item
    $\bbC$代数の反線型準同型写像
    \begin{equation}
      \End_\bbC(V)\lra\End_\bbC(\overline{V});
      f \lmt f \otimes_\sigma \id_\bbC
    \end{equation}
    を共役と呼ぶ。
    これは、$V$の基底とそれに伴う$\overline{V}$の基底のもとで、
    表現行列の各成分の複素共役をとる操作に対応する。
    \item
    自己準同型に対する転置と共役は可換な操作である。
    \item
    転置と共役の合成は$\bbC$代数の反線型反準同型写像
    $\dagger\colon\End_\bbC(V)\lra\End_\bbC(\overline{V}^{\vee})$
    であり、Hermitian共役と呼ばれる。
    これは$V$の基底とそれに伴う$\overline{V}^\vee$の基底のもとで、
    表現行列のHermitian共役をとる操作に対応する。
  \end{enumerate}
\end{dfn}

\begin{dfn}
  $(\rho, \Delta)$を
  $\Cl^{1+3}(\bbR)$の複素表現とするとき、
  表現$(\overline{\rho}^\vee,\overline{\Delta}^\vee)$を
  合成
  \begin{equation}
    \Cl^{1+3}(\bbR)\stackrel{\beta}{\lra}\Cl^{1+3}(\bbR)
    \stackrel{\rho}{\lra}
    \End_\bbC(\Delta)\stackrel{\dagger}{\lra}\End_\bbC(\overline{\Delta}^\vee)
  \end{equation}
  により定める。
  ここで、$\bbR$代数の反準同型写像
  $\beta\colon\Cl^{1+3}(\bbR)\lra\Cl^{1+3}(\bbR)$は
  定義\ref{dfn-beta-tranposing-antihomomorphism-clifford-algebra}のものである。
\end{dfn}

\begin{dfn}
  $\Cl^{1+3}(\bbR)$のDirac表現$\Delta$とは、
  Hermitian内積$\langle,\rangle$を持つ$\bbC$ベクトル空間$\Delta$上の既約表現であって、
  $\bbC$線型写像
  \begin{equation}
    \Delta \stackrel{e_0}{\lra} \Delta
    \stackrel{v \lmt \langle \cdot, v \rangle}{\lra}
    \overline{\Delta}^\vee
  \end{equation}
  が表現の準同型になっているもののことをいう。
  Dirac表現に対し、上の写像を$\psi\lmt\overline{\psi}$と書き、Dirac共役と呼ぶ。
\end{dfn}

\begin{thm}
  $\Delta=\bbC^4$に標準的なHermitian内積を定める時、
  ガンマ行列による表現
  $\gamma\colon\Cl^{1+3}(\bbR)\lra \Mat_4(\bbC)$
  はDirac表現である。
\end{thm}

\begin{proof}
  $\Delta$および$\overline{\Delta}^\vee$の標準基底に関して、写像
  $\Delta\stackrel{v \lmt \langle \cdot, v\rangle}{\lra}\overline{\Delta}^\vee$
  の表現行列は単位行列であるため、
  合成
  \begin{equation}
    \Delta \stackrel{e_0}{\lra} \Delta
    \stackrel{v \lmt \langle \cdot, v \rangle}{\lra}
    \overline{\Delta}^\vee
  \end{equation}
  の表現行列は$\gamma^0$となる。
  よって、ガンマ行列の関係式$(\gamma^\mu)^\dagger\gamma^0=\gamma^0\gamma^\mu$
  よりこの写像が表現の同型であることがわかる。
\end{proof}

\begin{thm}
  Clifford代数$\Cl^{1+3}(\bbR)$の既約な複素表現$\Delta$に対し、
  $\Delta$をDirac表現たらしめるようなHermitian内積$\langle,\rangle$が
  スカラー倍を除いて一意に存在する。
\end{thm}

\begin{proof}
  Clifford代数$\Cl^{1+3}(\bbR)$の既約表現の分類と、Schurの補題より従う。
\end{proof}
